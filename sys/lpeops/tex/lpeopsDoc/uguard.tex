\chapter{uguard}

\section{Introduction}

\txs{} checks whether or not there is \emph{input-output-conformance} of an implementation with the \emph{underspecified} traces of its specification (u-ioco, see \cite{volpato2013towards}).
In this context, the specification has only to be u-ioco with itself, and it can therefore potentially be rewritten more aggressively than when a models must preserve a stronger equivalence relation.

Unfortunately, there is no information about the traces of a model that can be extracted directly from that model while it is in LPE form.
The \texttt{uguard} command is an attempt to symbolically detect a specific, relatively simple pattern with actions that do not follow the u-ioco definition.
Upon detection, the command adds guards to the LPE to exclude the underspecified actions from the model (hence its name).

\section{Formal background}

\subsection{Summand implication}

Consider two summands, $s_\alpha$ and $s_\beta$, and reference their elements conform \ref{summandelements}.

Summand $s_\alpha$ is said to \emph{imply} $s_\beta$ if these two conditions hold:

\begin{itemize}
\item $s_\alpha$ and $s_\beta$ must communicate over the exact same channel with exactly as many channel variables; that is, $C_\alpha = C_\beta \land m_\alpha = m_\beta$.

\item Define the mapping
\begin{align*}
X_\beta = [x_\alpha(j) \rightarrow x_\beta(j) \;|\; 1 \leq j \leq \text{min}(m_1, m_2)]
\end{align*}

The following condition must be a tautology:
\begin{align*}
g_\alpha[X_\beta] \rightarrow g_\beta
\end{align*}
\end{itemize}

\subsection{Possible input-successors}

Consider a summand pair $(s_\alpha, s_\beta)$, referencing the elements of $s_\alpha$ and $s_\beta$ conform \ref{summandelements}.
Summand $s_\beta$ is said to be a \emph{possible input-successor} of $s_\alpha$ if $C_\beta$ is an input channel and if the following expression is satisfiable:
\begin{align*}
g_\alpha \land {g_\beta}[v \rightarrow q(v) \;|\; v \in \varsof{g_\beta} \setminus P][p \rightarrow v_\alpha(p) \;|\; p \in P]
\end{align*}

where $q(v)$ is a bijective function that relates variable $v$ to a fresh variable.

\subsection{Definite input-successors}

Consider a summand pair $(s_\alpha, s_\beta)$, referencing the elements of $s_\alpha$ and $s_\beta$ conform \ref{summandelements}.
Summand $s_\beta$ is said to be a \emph{definite input-successor} of $s_\alpha$ if $C_\beta$ is an input channel and if the following expression is a tautology:
\begin{align*}
g_\alpha \rightarrow {g_\beta}[v \mapsto q(v) \;|\; v \in \varsof{g_\beta} \setminus P][p \mapsto v_\alpha(p) \;|\; p \in P]
\end{align*}

where $q(v)$ is a bijective function that relates variable $v$ to a fresh variable.

\section{Algorithm}

Let the input LPE model be $M$ and the underlying LPE be $P$.
The \texttt{uguard} algorithm follows these steps for all summand pairs $(s_\alpha, s_\beta)$ such that $s_\alpha$ implies $s_\beta$:

\begin{enumerate}
\item Compute $\Delta = (D_\beta \setminus P_\alpha) \cup (D_\alpha \setminus P_\beta)$ where
\begin{itemize}
\item $P_\alpha$ is the set of all possible input-successors of $s_\alpha$;
\item $P_\beta$ is the set of all possible input-successors of $s_\beta$;
\item $D_\alpha$ is the set of all definite input-successors of $s_\alpha$; and
\item $D_\beta$ is the set of all definite input-successors of $s_\beta$.
\end{itemize}

If $\Delta = \emptyset{}$, the \texttt{uguard} algorithm does not make any changes and stops.
Otherwise, it continues.

\item Create a new LPE $P'$ with the same parameters as $P$, but without any summands.
Create a new LPE model $M'$ that has the same definition as $M$, except that its references to $P$ have been replaced by references to $P'$.

\item Add a new, fresh parameter $f$ of type \texttt{Bool} to $P'$.
Where $M'$ instantiates $P'$, initialize $f$ with $\textbf{true}$.

\item For each summand $s_i$ of $P$ -- referencing its elements conform \ref{summandelements} -- add to $P'$ a new summand ${s_i}'$ that is defined as
\begin{align*}
{s_i}' = C_i \; \texttt{?} \; x_i(1) \; &\cdots{} \; \texttt{?} \; x_i(m_i) \; [[\Gamma(s_i) \land g_i]] \\
&\texttt{>->} \; P(v_i(p_1), \cdots{}, v_i(p_k), \Upsilon(s_i))
\end{align*}

where
\begin{align*}
\Gamma(s) = \begin{cases}
f \text{ if } s \in \Delta \\
\textbf{true} \text{ if } s \notin \Delta
\end{cases}
\end{align*}

and
\begin{align*}
\Upsilon(s) = \begin{cases}
\textbf{true} \text{ if } s \neq s_\alpha \land s \neq s_\beta \\
\textbf{false} \text{ if } s = s_\alpha \lor s = s_\beta
\end{cases}
\end{align*}
\end{enumerate}

Intuitively, the algorithm first detects that certain summands are underspecified: they are enabled after a certain action if that action is performed by summand $s_\alpha$, but they are not enabled after that action is performed by summand $s_\beta$.
The algorithm therefore makes it so that the underspecified summands are always disabled after $s_\alpha$ or $s_\beta$ (preserving equivalence up to u-ioco).

