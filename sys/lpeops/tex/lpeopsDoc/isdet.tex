\chapter{isdet} \label{isdet}

\section{Introduction}

The \texttt{isdet} command checks whether the specified LPE is deterministic.
The command may yield false negatives.

\section{Formal background}

\subsection{Summand determinism}

Consider two summands, $s_\alpha$ and $s_\beta$, and reference their elements conform \ref{summandelements}.

Summands $s_\alpha$ and $s_\beta$ are said to be \emph{deterministic} if one of these conditions holds:

\begin{itemize}
\item $s_\alpha$ and $s_\beta$ are the same summand; that is, $s_\alpha = s_\beta$.

\item $s_\alpha$ and $s_\beta$ communicate over different channels:
\begin{align*}
C_\alpha \neq C_\beta \land (C_\alpha \in \{\istep{}, \cistep{}\} \not\leftrightarrow C_\beta \in \{\istep{}, \cistep{}\})
\end{align*}

\item $s_\alpha$ and $s_\beta$ communicate with different numbers of communication variables; that is, $m_\alpha \neq m_\beta$.

\item $s_\alpha$ and $s_\beta$ are never simultaneously enabled, or $s_\alpha$ and $s_\beta$ always lead to the same next state.
In order to determine this, check if
\begin{align*}
g_\alpha[X_\beta] \land g_\beta \rightarrow \bigwedge\limits_{j=1}^{k} v_\alpha(p_j)[X_\beta] = v_\beta(p_j)
\end{align*}

is a tautology, where $X_\beta$ is defined as
\begin{align*}
X_{\beta} &= [x_\alpha(j) \rightarrow x_\beta(j) \;|\; 1 \leq j \leq \text{min}(m_\alpha, m_\beta)]
\end{align*}
\end{itemize}

Note that this approach is \emph{not} guaranteed to correctly recognize that two summands are deterministic (false negatives are tolerated)!

\section{Algorithm}

The algorithm invoked by the \texttt{isdet} command checks for all pairs of different summands whether the first summand is deterministic with the second summand (see the previous section).
The algorithm yields \textbf{true} if and only if this is the case for all summand pairs.

\section{Example}

Consider the following LPE:

\begin{lstlisting}
//Process definition:
PROCDEF example[A :: Int](x, y :: Int)
  = A ? i [[x==0 /\ i>=0 /\ i<=5]] >-> example[A](1, 0)
  + A ? i [[y==0 /\ i>=5 /\ i<=9]] >-> example[A](0, 1)
  ;

//Initialization:
example[A](0, 1);
\end{lstlisting}

Consider the two summands, calling the first $s_1$ and the second $s_2$.

The first three conditions for detecting determinism are false.

For the fourth condition, the antecedent is
\begin{align*}
g_1[X_2] \land g_2 &\Leftrightarrow (\texttt{x} = 0 \land \texttt{i} \geq 0 \land \texttt{i} \leq 5)[\texttt{i} \rightarrow \texttt{i}] \land (\texttt{y} = 0 \land \texttt{i} \geq 5 \land \texttt{i} \leq 9) \\
&\Leftrightarrow \texttt{x} = 0 \land \texttt{y} = 0 \land \texttt{i} = 5
\end{align*}

Given the antecedent, the conclusion must hold.
This is not the case:
\begin{align*}
v_1(\texttt{x})[X_2] = 1[\texttt{i} \rightarrow \texttt{i}] = 1 \neq 0 = v_2(\texttt{x}) \\
v_1(\texttt{y})[X_2] = 0[\texttt{i} \rightarrow \texttt{i}] = 0 \neq 1 = v_2(\texttt{y}) \\
\end{align*}

$s_1$ is therefore \emph{not} deterministic with $s_2$.

Changing the LPE to

\begin{lstlisting}
//Process definition:
PROCDEF example[A :: Int](x, y :: Int)
  = A ? i [[x==0 /\ i>=0 /\ i<=4]] >-> example[A](1, 0)
  + A ? i [[y==0 /\ i>=5 /\ i<=9]] >-> example[A](0, 1)
  ;

//Initialization:
example[A](0, 1);
\end{lstlisting}

will result in an antecedent that is false; therefore, $s_1$ is deterministic with $s_2$.

